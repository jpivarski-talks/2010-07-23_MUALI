\documentclass[compress]{beamer}
\usepackage{ifthen,verbatim}

\newcommand{\isnote}{}
\xdefinecolor{lightyellow}{rgb}{1.,1.,0.25}
\xdefinecolor{darkblue}{rgb}{0.1,0.1,0.7}

%% Uncomment this to get annotations
%% \def\notes{\addtocounter{page}{-1}
%%            \renewcommand{\isnote}{*}
%% 	   \beamertemplateshadingbackground{lightyellow}{white}
%%            \begin{frame}
%%            \frametitle{Notes for the previous page (page \insertpagenumber)}
%%            \itemize}
%% \def\endnotes{\enditemize
%% 	      \end{frame}
%%               \beamertemplateshadingbackground{white}{white}
%%               \renewcommand{\isnote}{}}

%% Uncomment this to not get annotations
\def\notes{\comment}
\def\endnotes{\endcomment}

\setbeamertemplate{navigation symbols}{}
\setbeamertemplate{headline}{\mbox{ } \hfill
\begin{minipage}{5.5 cm}
\vspace{-0.75 cm} \small
\end{minipage} \hfill
\begin{minipage}{4.5 cm}
\vspace{-0.75 cm} \small
\begin{flushright}
\ifthenelse{\equal{\insertpagenumber}{1}}{}{Jim Pivarski \hspace{0.2 cm} \insertpagenumber\isnote/\pageref{numpages}}
\end{flushright}
\end{minipage}\mbox{\hspace{0.2 cm}}\includegraphics[height=1 cm]{../cmslogo} \hspace{0.01 cm} \vspace{-1.05 cm}}

\begin{document}
\begin{frame}
\vfill
\begin{center}
\textcolor{darkblue}{\Large Common Alignment Working Point Proposal}

\vfill
\begin{columns}
\column{0.75\linewidth}
\begin{center}
\large
Jim Pivarski
\end{center}
\end{columns}

\vfill
23 July, 2010

\end{center}
\end{frame}

%% \begin{notes}
%% \item This is the annotated version of my talk.
%% \item If you want the version that I am presenting, download the one
%% labeled ``slides'' on Indico (or just ignore these yellow pages).
%% \item The annotated version is provided for extra detail and a written
%% record of comments that I intend to make orally.
%% \item Yellow notes refer to the content on the {\it previous} page.
%% \item All other slides are identical for the two versions.
%% \end{notes}

\small

\begin{frame}
\frametitle{Motivation}
\begin{itemize}
\item In mid-August, we will be asked for a new alignment (very important to provide one, because the current geometry has known issues)
\item We may not have solved the track-based/hardware discrepancies
\item We must provide a best-knowledge geometry anyway, and make choices in cases where our knowledge is incomplete
\end{itemize}

\vfill
\hspace{-0.83 cm} \textcolor{darkblue}{\Large Proposal}
\begin{itemize}
\item Sasha's GlobalPositionRcd (TB \& HW agree)
\item Barrel:
\begin{itemize}
\item local-$y$ chamber positions from TB (clear discontinuities in residuals distribution)
\item re-run HW alignment with these local-$y$ positions to get a consistent set of local-$x$ (and other parameters)\textcolor{red}{\large *}
\end{itemize}

\item Endcap:
\begin{itemize}
\item chamber positions from beam-halo
\item combined fit with DCOPS {\it if consistent}\textcolor{red}{\large *}
\item disk positions using global cosmic tracks
\end{itemize}
\end{itemize}

\textcolor{red}{\large *} New, but possible on a short time-scale?
\label{numpages}
\end{frame}


%% \begin{frame}
%% \frametitle{Outline}
%% \begin{itemize}\setlength{\itemsep}{0.75 cm}
%% \item 
%% \end{itemize}
%% %% \hspace{-0.83 cm} \textcolor{darkblue}{\Large Outline2}
%% \end{frame}

%% \section*{First section}
%% \begin{frame}
%% \begin{center}
%% \Huge \textcolor{blue}{First section}
%% \end{center}
%% \end{frame}

%% \begin{frame}
%% \end{frame}

\end{document}
